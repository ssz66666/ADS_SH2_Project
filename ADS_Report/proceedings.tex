\documentclass{sigchi}

% Use this section to set the ACM copyright statement (e.g. for
% preprints).  Consult the conference website for the camera-ready
% copyright statement.

% Copyright
% \CopyrightYear{2016}
%\setcopyright{acmcopyright}
% \setcopyright{acmlicensed}
%\setcopyright{rightsretained}
%\setcopyright{usgov}
%\setcopyright{usgovmixed}
%\setcopyright{cagov}
%\setcopyright{cagovmixed}
% DOI
% \doi{http://dx.doi.org/10.475/123_4}
% ISBN
% \isbn{123-4567-24-567/08/06}
%Conference
% \conferenceinfo{CHI'16,}{May 07--12, 2016, San Jose, CA, USA}
%Price
% \acmPrice{\$15.00}

% \CopyrightYear{2016}
% \setcopyright{acmlicensed}
% \doi{http://dx.doi.org/10.475/123_4}
% \isbn{123-4567-24-567/08/06}
% \conferenceinfo{CHI'16,}{May 07--12, 2016, San Jose, CA, USA}
% \acmPrice{\$15.00}

% Use this command to override the default ACM copyright statement
% (e.g. for preprints).  Consult the conference website for the
% camera-ready copyright statement.

%% HOW TO OVERRIDE THE DEFAULT COPYRIGHT STRIP --
%% Please note you need to make sure the copy for your specific
%% license is used here!
% \toappear{
% Permission to make digital or hard copies of all or part of this work
% for personal or classroom use is granted without fee provided that
% copies are not made or distributed for profit or commercial advantage
% and that copies bear this notice and the full citation on the first
% page. Copyrights for components of this work owned by others than ACM
% must be honored. Abstracting with credit is permitted. To copy
% otherwise, or republish, to post on servers or to redistribute to
% lists, requires prior specific permission and/or a fee. Request
% permissions from \href{mailto:Permissions@acm.org}{Permissions@acm.org}. \\
% \emph{CHI '16},  May 07--12, 2016, San Jose, CA, USA \\
% ACM xxx-x-xxxx-xxxx-x/xx/xx\ldots \$15.00 \\
% DOI: \url{http://dx.doi.org/xx.xxxx/xxxxxxx.xxxxxxx}
% }

% Arabic page numbers for submission.  Remove this line to eliminate
% page numbers for the camera ready copy
\pagenumbering{arabic}

% Load basic packages
\usepackage{balance}       % to better equalize the last page
\usepackage{graphics}      % for EPS, load graphicx instead
\usepackage[T1]{fontenc}   % for umlauts and other diaeresis
\usepackage{txfonts}
\usepackage{mathptmx}
\usepackage[pdflang={en-US},pdftex]{hyperref}
\usepackage{color}
\usepackage{booktabs}
\usepackage{textcomp}
\usepackage{multirow}

% Some optional stuff you might like/need.
\usepackage{microtype}        % Improved Tracking and Kerning
% \usepackage[all]{hypcap}    % Fixes bug in hyperref caption linking
\usepackage{ccicons}          % Cite your images correctly!
% \usepackage[utf8]{inputenc} % for a UTF8 editor only

% If you want to use todo notes, marginpars etc. during creation of
% your draft document, you have to enable the "chi_draft" option for
% the document class. To do this, change the very first line to:
% "\documentclass[chi_draft]{sigchi}". You can then place todo notes
% by using the "\todo{...}"  command. Make sure to disable the draft
% option again before submitting your final document.
\usepackage{todonotes}

% Paper metadata (use plain text, for PDF inclusion and later
% re-using, if desired).  Use \emtpyauthor when submitting for review
% so you remain anonymous.
\def\plaintitle{}
\def\plainauthor{First Author, Second Author, Third Author,
  Fourth Author, Fifth Author, Sixth Author}
\def\emptyauthor{}
\def\plainkeywords{Authors' choice; of terms; separated; by
  semicolons; include commas, within terms only; required.}
\def\plaingeneralterms{Documentation, Standardization}

% llt: Define a global style for URLs, rather that the default one
\makeatletter
\def\url@leostyle{%
  \@ifundefined{selectfont}{
    \def\UrlFont{\sf}
  }{
    \def\UrlFont{\small\bf\ttfamily}
  }}
\makeatother
\urlstyle{leo}

% To make various LaTeX processors do the right thing with page size.
\def\pprw{8.5in}
\def\pprh{11in}
\special{papersize=\pprw,\pprh}
\setlength{\paperwidth}{\pprw}
\setlength{\paperheight}{\pprh}
\setlength{\pdfpagewidth}{\pprw}
\setlength{\pdfpageheight}{\pprh}

% Make sure hyperref comes last of your loaded packages, to give it a
% fighting chance of not being over-written, since its job is to
% redefine many LaTeX commands.
\definecolor{linkColor}{RGB}{6,125,233}
\hypersetup{%
  pdftitle={\plaintitle},
% Use \plainauthor for final version.
%  pdfauthor={\plainauthor},
  pdfauthor={\emptyauthor},
  pdfkeywords={\plainkeywords},
  pdfdisplaydoctitle=true, % For Accessibility
  bookmarksnumbered,
  pdfstartview={FitH},
  colorlinks,
  citecolor=black,
  filecolor=black,
  linkcolor=black,
  urlcolor=linkColor,
  breaklinks=true,
  hypertexnames=false
}

% create a shortcut to typeset table headings
% \newcommand\tabhead[1]{\small\textbf{#1}}

% End of preamble. Here it comes the document.
\makeatletter
\def\@copyrightspace{\relax}
\makeatother

\begin{document}

\title{\plaintitle}

\numberofauthors{3}
% \author{%
%   \alignauthor{Siyu Han\\
%     \affaddr{University of Bristol}\\
%     \affaddr{Bristol, UK}\\
%     \email{ec18242@bristol.ac.uk}}\\
% }

\maketitle

% \begin{abstract}
%   UPDATED---\today. This sample paper describes the formatting
%   requirements for SIGCHI conference proceedings, and offers
%   recommendations on writing for the worldwide SIGCHI
%   readership. Please review this document even if you have submitted
%   to SIGCHI conferences before, as some format details have changed
%   relative to previous years. Abstracts should be about 150 words and
%   are required.
% \end{abstract}
%
% \category{H.5.m.}{Information Interfaces and Presentation
%   (e.g. HCI)}{Miscellaneous} \category{See
%   \url{http://acm.org/about/class/1998/} for the full list of ACM
%   classifiers. This section is required.}{}{}
%
% \keywords{\plainkeywords}

Raw Bias (RBias), has a similar form Mean Absolute Error (MAE):
\begin{equation}
    \text{RBias} = \frac{1}{n}\sum_{i=1}^{n}|\hat{Q} - Q}|
\end{equation}

Percent Bias (PBias):
\begin{equation}
    \text{PBias} = \frac{1}{n}\sum_{i=1}^{n}|(\hat{Q} - Q})/Q| \times 100
\end{equation}


Root Mean Square Error (RMSE)
\begin{equation}
    \text{RMSE} = \sqrt{\frac{1}{n}\sum_{i=1}^{n}{(\hat{Q} - Q})^2}}
\end{equation}
where $Q$ is missing values, $\hat{Q}$ is the corresponding predicted values, and $n$ is the count of missing values.


\begin{table}[t]
    \centering
    \begin{tabular}{lrrrrrr}\toprule
        \multirow{2}{*}{\textbf{Subject}} & \multicolumn{2}{c}{\textbf{IMU hand}} & \multicolumn{2}{c}{\textbf{IMU chest}} & \multicolumn{2}{c}{\textbf{IMU ankle}}\\
        \cmidrule(r){2-3} \cmidrule(r){4-5} \cmidrule(r){6-7}
        & \emph{Count} & \emph{\%} & \emph{Count} & \emph{\%} & \emph{Count} & \emph{\%} \\ \midrule
        1 & 1454 & 0.39 & 509 & 0.14 & 1327 & 0.35 \\
        2 & 2729 & 0.61 & 387 & 0.09 & 2445 & 0.55 \\
        3 & 522  & 0.21 & 183 & 0.07 & 527  & 0.21 \\
        4 & 2214 & 0.67 & 213 & 0.06 & 1101 & 0.33 \\
        5 & 1541 & 0.41 & 312 & 0.08 & 1980 & 0.53 \\
        6 & 1021 & 0.28 & 343 & 0.09 & 1372 & 0.38 \\
        7 & 1506 & 0.48 & 257 & 0.08 & 1037 & 0.33 \\
        8 & 2151 & 0.53 & 1308 & 0.32 & 1951 & 0.48 \\ \bottomrule
    \end{tabular}
    \caption{The count and percentage of missing values in PAMAP2.}~\label{NA_PAMAP2}
\end{table}

\begin{table*}[t]
    \centering
    \begin{tabular}{lrrrrrrrr}\toprule
        \multirow{2}{*}{\textbf{Subject}} & \multicolumn{2}{c}{\textbf{Mean}} & \multicolumn{2}{c}{\textbf{Sample}} & \multicolumn{2}{c}{\textbf{LOCF}} & \multicolumn{2}{c}{\textbf{PMM}}\\
        \cmidrule(r){2-3} \cmidrule(r){4-5} \cmidrule(r){6-7} \cmidrule(r){8-9}
        & \emph{RBias} & \emph{RMSE} & \emph{RBias} & \emph{RMSE} & \emph{RBias} & \emph{RMSE} & \emph{RBias} & \emph{RMSE} \\ \midrule
        1 &   7.949 & 15.723 &  11.163 & 22.291 &	0.431   &   1.737	& 6.231 & 13.144 \\
        2 &   5.238 &  9.932 &   7.422 & 14.038 &	0.338	&	1.315	& 4.171 & 8.593 \\
        3 &   4.757 &  8.909 &   6.709 & 12.623 &	0.289	&	0.991	& 3.605 & 7.537 \\
        4 &   5.367 & 10.467 &   7.633 & 14.759 &	0.299	&	0.856	& 4.192 & 8.720 \\
        5 &   5.457 & 10.692 &   7.783 & 15.119 &	0.387	&	1.471	& 4.637 & 9.510 \\
        6 &   5.438 & 10.286 &   7.683 & 14.503 &	0.432	&	1.923	& 4.555 & 9.121 \\
        7 &   5.396 & 10.391 &   7.657 & 14.735 &	0.372	&	1.764	& 3.919 & 8.256 \\
        8 &   5.315 & 10.091 &   7.537 & 14.288 &	0.377	&	1.741	& 4.094 & 8.223 \\ \bottomrule
    \end{tabular}
    \caption{Evaluation of different imputation methods on PAMAP2 dataset.}~\label{EvaluationImputation}
\end{table}

\begin{equation}
    X_{e}[k] = \sum_{n = 0}^{N - 1}{X_{e}[n] e^{-j\frac{2\pi kn}{N}}} , \quad
    S_{e}[k] = |X_{e}[k]|^{2} .
\end{equation}

\begin{equation}
    \overline{E} = \frac{\sum_{k = 0}^{N - 1}{S_{e}[k]}}{N} ,\quad
    \text{SNR} = \frac{S_{e}[\frac{N}{3}]}{\overline{E}} .
% \vspace*{-8pt}
\end{equation}

% Balancing columns in a ref list is a bit of a pain because you
% either use a hack like flushend or balance, or manually insert
% a column break.  http://www.tex.ac.uk/cgi-bin/texfaq2html?label=balance
% multicols doesn't work because we're already in two-column mode,
% and flushend isn't awesome, so I choose balance.  See this
% for more info: http://cs.brown.edu/system/software/latex/doc/balance.pdf
%
% Note that in a perfect world balance wants to be in the first
% column of the last page.
%
% If balance doesn't work for you, you can remove that and
% hard-code a column break into the bbl file right before you
% submit:
%
% http://stackoverflow.com/questions/2149854/how-to-manually-equalize-columns-
% in-an-ieee-paper-if-using-bibtex
%
% Or, just remove \balance and give up on balancing the last page.
%
%\balance{}

% \section{References Format}


% BALANCE COLUMNS
%\balance{}

% REFERENCES FORMAT
% References must be the same font size as other body text.
% \bibliographystyle{SIGCHI-Reference-Format}
% \bibliography{sample}

% \section{Gannt Chart}

% \section{Risk Analysis}

\end{document}

%%% Local Variables:
%%% mode: latex
%%% TeX-master: t
%%% End:
